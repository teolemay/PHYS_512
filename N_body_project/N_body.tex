\documentclass{article}

%opening
\title{PHYS 512: N body simulation project}
\author{Teophile Lemay}
\date{}
\usepackage[left=2cm, right=2cm, top=2cm]{geometry}
\usepackage{amsmath}
\usepackage{amssymb}
\usepackage{physics}
\usepackage{graphicx}
\usepackage{listings}
\usepackage{xcolor}
\newcommand{\<}[1]{\left\langle #1 \right\rangle }

\definecolor{codegreen}{rgb}{0,0.6,0}
\definecolor{codegray}{rgb}{0.5,0.5,0.5}
\definecolor{codepurple}{rgb}{0.58,0,0.82}
\definecolor{backcolour}{rgb}{0.95,0.95,0.92}

\lstdefinestyle{mystyle}{
	backgroundcolor=\color{backcolour},   
	commentstyle=\color{codegreen},
	keywordstyle=\color{magenta},
	numberstyle=\tiny\color{codegray},
	stringstyle=\color{codepurple},
	basicstyle=\ttfamily\footnotesize,
	breakatwhitespace=false,         
	breaklines=true,                 
	captionpos=b,                    
	keepspaces=true,                 
	numbers=left,                    
	numbersep=5pt,                  
	showspaces=false,                
	showstringspaces=false,
	showtabs=false,                  
	tabsize=2
}

\lstset{style=mystyle}


\begin{document}
\maketitle

The code used to run the N-body simulation is strongly inspired by, and adapted from the code provided in \texttt{nbody\_nb\_fft.py}. In particular, the single body potential field generation, and the gradient calculation functions are directly copied. My N-body simulation code can be found in \texttt{Nbody.py}.

\section{Part 1: Single particle}
Figure 1 shows the time evolution of the x and y positions of a single particle with zero initial velocity. As expected, there is no evolution and the particle does not move since there are no external forces acting upon it.
\begin{figure}[h]
	\caption{Single particle stays stationary}
	\includegraphics[scale=0.5]{1particlex}
	\includegraphics[scale=0.5]{1particley}
\end{figure} 

\section{Part 2: 2 particle orbit}
Figure 2 shows the time evolution of the x and y positions of two particles in a circular orbit. Running the simulation with a 0.01 second time step for 50000 steps, both particles maintain a stable orbit. Though the orbits are fairly stable, they are not perfect, as shown by the slow overall displacement in the positive $y$ direction. This irregularity may be due to imperfect initial conditions for the orbit linked to limitations in the resolution of the potential grid.
\begin{figure}[h]
	\caption{Two particles in orbit}
	\includegraphics[scale=0.5]{xorbit}
	\includegraphics[scale=0.5]{yorbit}
\end{figure}
	
\section{Part 3: Energy conservation}
For leapfrog integrator to conserve energy, must guarantee the CFL condition.



	
\end{document}
